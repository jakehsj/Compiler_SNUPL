\documentclass[12pt]{article}
\usepackage[paper=letterpaper,margin=2cm]{geometry}
\usepackage{amsmath}
\usepackage{amssymb}
\usepackage{amsfonts}
\usepackage{newtxtext, newtxmath}
\usepackage{enumitem}
\usepackage{titling}
\usepackage[colorlinks=true]{hyperref}
\usepackage{simplebnf}
\usepackage{tabularray}
\usepackage{stackrel}
\usepackage{mathtools}
\usepackage{authblk}
\usepackage{ebproof}
\usepackage{kotex}
\usepackage{algorithm,algorithmic, }
\usepackage{listings}
\usepackage{xcolor}
\usepackage{caption}

\hbadness=99999
\ebproofset{left label template=\textsc{[\inserttext]}}

\newlength\myindent
\setlength\myindent{2em}
\newcommand\bindent{%
  \begingroup
  \setlength{\itemindent}{\myindent}
  \addtolength{\algorithmicindent}{\myindent}
}
\newcommand\eindent{\endgroup}

\renewcommand{\algorithmicrequire}{\textbf{Input:}}
\renewcommand{\algorithmicensure}{\textbf{Output:}}
    

\definecolor{codegreen}{rgb}{0,0.6,0}
\definecolor{codegray}{rgb}{0.5,0.5,0.5}
\definecolor{codepurple}{rgb}{0.58,0,0.82}
\definecolor{backcolour}{rgb}{0.95,0.95,0.92}

\lstdefinestyle{mystyle}{
    backgroundcolor=\color{backcolour},   
    commentstyle=\color{codegreen},
    keywordstyle=\color{magenta},
    numberstyle=\tiny\color{codegray},
    stringstyle=\color{codepurple},
    basicstyle=\ttfamily\footnotesize,
    breakatwhitespace=false,         
    breaklines=true,                 
    captionpos=b,                    
    keepspaces=true,                 
    numbers=left,                    
    numbersep=5pt,                  
    showspaces=false,                
    showstringspaces=false,
    showtabs=false,                  
    tabsize=2
}

\lstset{style=mystyle}

\title{Syantx Analysis}

\author{Oh Gyuhyeok (2020-10485)}
\author{Hwang SeungJun (2020-15313)}
\affil{SNUCSE}
\date{\today}

\begin{document}
\maketitle

\begin{abstract}
    abstract
\end{abstract}

\section{Introduction}

In the second phase of our term project, our primary focus was on implementing a top-down parser for the SnuPL/2 language. 
This report provides an overview of the work completed during this phase, including the design, implementation, and testing of the parser.
cite \cite[cite1]{Textbook_Compiler}
cite \cite[cite2]{Egger_Lecture_Note}
cite \cite[cite3]{Egger_git}


\section{Progress Overview}
The main objectives of this phase were as follows:\\
1. Design and implement a top-down parser for the SnuPL/2 language. \\
2. Generate an abstract syntax tree (AST) in both textual and graphical forms. \\
3. Perform syntactical checks, such as ensuring symbols are defined before use. \\
4. Enforce constraints related to the SnuPL/2 syntax, such as matching declaration and end identifiers. \\

\section{Design}
\subsection{Grammar}
EBNF notation is used to describe the grammar of the SnuPL/2 language.
% TODO: translate to form fitting to latex.
% module            = "module" ident ";" 
% { constDeclaration | varDeclaration | subroutineDecl }
% [ "begin" statSequence ] "end" ident ".".

% letter            = "A".."Z" | "a".."z" | "_".
% digit             = "0".."9".
% hexdigit          = digit | "A".."F" | "a".."f".
% character         = LATIN1_char | "\n" | "\t" | "\"" | "\'" | "\\" | hexencoded.
% hexedcoded        = "\x" hexdigit hexdigit.
% char              = "'" character  "'" | "'" "\0" "'".
% string            = '"' { character } '"'.

% ident             = letter { letter | digit }.
% number            = digit { digit } [ "L" ].
% boolean           = "true" | "false".
% type              = basetype | type "[" [ simpleexpr ] "]".
% basetype          = "boolean" | "char" | "integer" | "longint".

% qualident         = ident { "[" simpleexpr "]" }.
% factOp            = "*" | "/" | "&&".
% termOp            = "+" | "-" | "||".
% relOp             = "=" | "#" | "<" | "<=" | ">" | ">=".

% factor            = qualident | number | boolean | char | string |
% "(" expression ")" | subroutineCall | "!" factor.
% term              = factor { factOp factor }.
% simpleexpr        = ["+"|"-"] term { termOp term }.
% expression        = simpleexpr [ relOp simplexpr ].

% assignment        = qualident ":=" expression.
% subroutineCall    = ident "(" [ expression {"," expression} ] ")".
% ifStatement       = "if" "(" expression ")" "then" statSequence
% [ "else" statSequence ] "end".
% whileStatement    = "while" "(" expression ")" "do" statSequence "end".
% returnStatement   = "return" [ expression ].

% statement         = assignment | subroutineCall | ifStatement
% | whileStatement | returnStatement.
% statSequence      = [ statement { ";" statement } ].

% constDeclaration  = [ "const" constDeclSequence ].
% constDeclSequence = constDecl ";" { constDecl ";" }
% constDecl         = varDecl "=" expression.

% varDeclaration    = [ "var" varDeclSequence ";" ].
% varDeclSequence   = varDecl { ";" varDecl }.
% varDecl           = ident { "," ident } ":" type.

% subroutineDecl    = (procedureDecl | functionDecl)
% ( "extern" | subroutineBody ident ) ";".
% procedureDecl     = "procedure" ident [ formalParam ] ";".
% functionDecl      = "function" ident [ formalParam ] ":" type ";".
% formalParam       = "(" [ varDeclSequence ] ")".
% subroutineBody    = constDeclaration varDeclaration
% "begin" statSequence "end".

% comment           = "//" {[^\n]} \n.
% whitespace        = { " " | \t | \n }.

\subsection{Method}
We implemented additional method to implement parser.
List of method is below.

\begin{itemize}
    \item \texttt{parseModule} : parse module
    \item \texttt{parseConstDecl} : parse const declaration
    \item \texttt{parseVarDecl} : parse variable declaration
    \item \texttt{parseSubroutineDecl} : parse subroutine declaration
    \item \texttt{parseFormalParam} : parse formal parameter
    \item \texttt{parseSubroutineBody} : parse subroutine body
    \item \texttt{parseStatSequence} : parse statement sequence
    \item \texttt{parseStatement} : parse statement
    \item \texttt{parseAssignment} : parse assignment
    \item \texttt{parseSubroutineCall} : parse subroutine call
    \item \texttt{parseIfStatement} : parse if statement
    \item \texttt{parseWhileStatement} : parse while statement
    \item \texttt{parseReturnStatement} : parse return statement
    \item \texttt{parseExpression} : parse expression
    \item \texttt{parseSimpleExpr} : parse simple expression
    \item \texttt{parseTerm} : parse term
    \item \texttt{parseFactor} : parse factor
    \item \texttt{parseQualident} : parse qualident
    \item \texttt{parseNumber} : parse number
    \item \texttt{parseBoolean} : parse boolean
    \item \texttt{parseChar} : parse char
    \item \texttt{parseString} : parse string
    \item \texttt{parseType} : parse type
    \item \texttt{parseBasetype} : parse basetype
    \item \texttt{parseIdent} : parse identifier
    \item \texttt{parseRelOp} : parse relational operator
    \item \texttt{parseTermOp} : parse term operator
    \item \texttt{parseFactOp} : parse factor operator
    \item \texttt{constChar} : parse constant char
\end{itemize}
\subsection{Algorithm}
\begin{algorithm}[H]
    \begin{algorithmic}
        \STATE\caption{Description of the algorithm}
        \REQUIRE\(\text{input}\)
        \ENSURE \(\text{output}\)
        \STATE\(AAA\)\\
    \end{algorithmic}
\end{algorithm}

\section{Result}
We utilized a provided test program that printed the AST to validate our parser's functionality.
The test program allowed us to examine the generated AST for different test cases.
In addition to the provided test cases, we created custom test cases to cover more complex and special scenarios, ensuring the robustness of our parser.

\bibliographystyle{plain}
\bibliography{report}

\newpage

\appendix
\addcontentsline{toc}{section}{Appendices}
\section{Appendix}
\subsection{Additional Test Cases}
\textbf{Input:}
\begin{lstlisting}[language=Python, caption={[short]caption text}]
    import numpy as np
        
    def incmatrix(genl1,genl2):
        m = len(genl1)
        n = len(genl2)
        M = None #to become the incidence matrix
        VT = np.zeros((n*m,1), int)  #dummy variable
        
        #compute the bitwise xor matrix
        M1 = bitxormatrix(genl1)
        M2 = np.triu(bitxormatrix(genl2),1) 
    
        for i in range(m-1):
            for j in range(i+1, m):
                [r,c] = np.where(M2 == M1[i,j])
                for k in range(len(r)):
                    VT[(i)*n + r[k]] = 1;
                    VT[(i)*n + c[k]] = 1;
                    VT[(j)*n + r[k]] = 1;
                    VT[(j)*n + c[k]] = 1;
                    
                    if M is None:
                        M = np.copy(VT)
                    else:
                        M = np.concatenate((M, VT), 1)
                    
                    VT = np.zeros((n*m,1), int)
        
        return M
\end{lstlisting}

\textbf{Output:}
\begin{verbatim}
    result
\end{verbatim}
\end{document}
